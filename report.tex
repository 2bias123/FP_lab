\documentclass[12pt,a4paper]{article}
\input{latexmacros.tex}

\title{Classical Cryptography}
\author{Tobias Innleggen, András Zsolt Sütő, Azim Kazimli}
\date{\today}
\hypersetup{pdfauthor={Tobias Innleggen, Andráas Zsolt Süutő, Azim Kazimli}, pdftitle={Classical Cryptography}}


\begin{document}

\maketitle


\begin{abstract}
    In this project, we investigate the effectiveness of using Haskell
     in implementing and breaking classical ciphers. Our report focuses on three historically significant ciphers: the Caesar cipher, 
    the Vigen\`ere cipher, and the Playfair cipher. We develop automated methods to recognize which cipher has been used on a given ciphertext by analyzing statistical 
    features such as letter frequencies, $n$-gram patterns, entropy, and periodicity. Based on the identified cipher, we then apply appropriate cryptanalytic attacks: 
    brute-force key search for Caesar and Playfair, and a combination of the Friedman test, Kasiski examination, and frequency analysis for Vigen\`ere. We evaluate the
     success rate of these attacks and the performance of our Haskell implementations. 
      \textbf{(Final results will be inserted here upon completion, summarizing the accuracy and efficiency achieved.)}
    \end{abstract}
\vfill

\tableofcontents

\clearpage

% Section 1 - Introduction

\section{Introduction}

The motivation behind this report is twofold. First, we aim to demonstrate that a functional language like Haskell can elegantly implement classical ciphers and their cryptanalysis. Second, we address a practical scenario: given an unknown piece of encrypted text, can we automatically determine the cipher used and subsequently crack it? 

First we implement three classical ciphers:

\begin{itemize}
    \item \textbf{Caesar cipher} is a simple monoalphabetic substitution cipher that shifts letters by a fixed offset; it was famously used by Julius Caesar and exemplifies the most basic substitution technique. 

    \item \textbf{Vigen\`ere cipher} extends this idea using a keyword to vary the shift across the message, making it a polyalphabetic cipher that was once considered "le chiffre ind\'echiffrable" (the indecipherable cipher) due to its resistance to simple frequency analysis.

    \item \textbf{Playfair cipher} is a digraph substitution cipher that encrypts pairs of letters using a 5x5 key matrix. This cypher mixes letter coordinates and thus complicates frequency patterns. \cite{banoth2023}
\end{itemize}

Caesar is trivial to break with brute force, Vigen\`ere requires analysis to uncover its repeating key, and Playfair demands strategic searching using a list of common words due to its larger key space. Both the ciphered text and the key are assumed to be in English.

We built a \textbf{cipher recognition} mechanism that inspects an unknown ciphertext and infers which cipher produced it. We base this recognition on statistical features of the ciphertext, such asletter frequency distributions, entropy and coincidence measures and periodicity, as explained in Section \ref{sec:recognition}.

We measure the success rate of our attacks by testing each cipher-breaking method on sample ciphertexts and checking if the correct plaintext (or key) is recovered. We also profile the performance (time complexity and efficiency) of the Haskell solutions. Another aspect of our evaluation is the \textit{ease of coding} and clarity of the solutions in Haskell. Throughout the project, we found that Haskell’s expressive features (like higher-order functions, list comprehensions, and immutable data structures) allowed us to write cryptographic algorithms in a concise manner.


% Section 2 - Cipher implemenation

\section{Implementing classical cryptography ciphers}\label{sec:implementation}

Some more theory, some more citing like \cite{lamport1994latex}.

\input{lib/Substitution.lhs}

\input{lib/Vigenere.lhs}

\input{lib/Playfair.lhs}

% Section 3 - Cipher recognition
\input{lib/CipherRecognition.lhs}

% Section 4 - Attacks
\section{Attacks}\label{sec:Main}

This section will have the implementation for the attacks on the ciphers.

\input{lib/SubstitutionBreaker.lhs}
\input{lib/VigenereBreaker.lhs}
\input{lib/PlayfairBreaker.lhs}

% Section 5 - Tests
\input{test/Tests.lhs}

% Section 6 - Conclusion

\section{Conclusion}\label{sec:Conclusion}

Here we conclude what has to be concluded.


\addcontentsline{toc}{section}{Bibliography}
\bibliographystyle{alpha}
\bibliography{references.bib}

\end{document}
